\documentclass[manuscript,screen,review]{acmart}
\documentclass{article}
\usepackage{listings}
\usepackage{tikz}
\usetikzlibrary{shapes, positioning}
\usepackage{caption}
\usepackage{subcaption}

\lstset{
	basicstyle=\ttfamily\footnotesize,
	frame=single,
	columns=flexible,
	keepspaces=false,
	showspaces=false,
	showstringspaces=false,
	showtabs=false,
	tabsize=1,
	breaklines=true,
	breakatwhitespace=true,
	numbers=none,
	numberstyle=\tiny,
	stepnumber=1,
	numbersep=5pt,
	backgroundcolor=\color{gray!10}, % Set background color
	keywordstyle=\color{blue}, % Color keywords
	commentstyle=\color{green}, % Color comments
	stringstyle=\color{red}, % Color strings
	morekeywords={__m256, _mm256_setzero_ps, _mm256_loadu_ps, _mm256_fmadd_ps, _mm256_storeu_ps}, % Additional keywords for AVX
	escapeinside={(*}{*)}
}

\newcommand{\framedarg}[1]{
	\tikz[baseline=(char.base)]
	\node[draw=blue, rectangle, inner sep=1pt, anchor=base] (char) {#1};
}

\begin{document}
	
	\begin{figure}[h]
		\centering
		\begin{subfigure}[t]{0.45\textwidth}
			\centering
			\caption{Naive Matrix-Vector Multiplication}
			\label{lst:naive}
			\begin{lstlisting}[language=C]
				void naive_mvm(float *(*\framedarg{A}*), float *(*\framedarg{x}*), float *(*\framedarg{y}*), int (*\framedarg{N}*)) {
					for (int i = 0; i < (*\framedarg{N}*); ++i) {
						(*\framedarg{y}*)[i] = 0;
						for (int j = 0; j < (*\framedarg{N}*); ++j) {
							(*\framedarg{y}*)[i] += (*\framedarg{A}*)[i * (*\framedarg{N}*) + j] * (*\framedarg{x}*)[j];
						}
					}
				}
			\end{lstlisting}
		\end{subfigure}
		\hspace{1em}
		\begin{subfigure}[t]{0.45\textwidth}
			\centering
			\caption{Unrolled Matrix-Vector Multiplication}
			\label{lst:unrolled}
			\begin{lstlisting}[language=C]
				void unrolled_mvm(float *(*\framedarg{A}*), float *(*\framedarg{x}*), float *(*\framedarg{y}*), int (*\framedarg{N}*)) {
					for (int i = 0; i < (*\framedarg{N}*); ++i) {
						(*\framedarg{y}*)[i] = 0;
						for (int j = 0; j < (*\framedarg{N}*); j += 4) {
							(*\framedarg{y}*)[i] += (*\framedarg{A}*)[i * (*\framedarg{N}*) + j] * (*\framedarg{x}*)[j];
							(*\framedarg{y}*)[i] += (*\framedarg{A}*)[i * (*\framedarg{N}*) + j + 1] * (*\framedarg{x}*)[j + 1];
							(*\framedarg{y}*)[i] += (*\framedarg{A}*)[i * (*\framedarg{N}*) + j + 2] * (*\framedarg{x}*)[j + 2];
							(*\framedarg{y}*)[i] += (*\framedarg{A}*)[i * (*\framedarg{N}*) + j + 3] * (*\framedarg{x}*)[j + 3];
						}
					}
				}
			\end{lstlisting}
		\end{subfigure}
		\vspace{1em}
		\begin{subfigure}[t]{0.45\textwidth}
			\centering
			\caption{AVX Intrinsics Matrix-Vector Multiplication}
			\label{lst:avx}
			\begin{lstlisting}[language=C++]
				#include <immintrin.h>
				
				void avx_mvm(float *(*\framedarg{A}*), float *(*\framedarg{x}*), float *(*\framedarg{y}*), float (*\framedarg{alpha}*), float (*\framedarg{beta}*), int (*\framedarg{lda}*), int (*\framedarg{N}*)) {
					for (int i = 0; i < (*\framedarg{N}*); ++i) {
						__m256 vec_y = _mm256_setzero_ps();
						for (int j = 0; j < (*\framedarg{N}*); j += 8) {
							__m256 vec_a = _mm256_loadu_ps(&(*\framedarg{A}*)[i * (*\framedarg{lda}*) + j]);
							__m256 vec_x = _mm256_loadu_ps(&(*\framedarg{x}*)[j]);
							vec_y = _mm256_fmadd_ps(vec_a, vec_x, vec_y);
						}
						(*\framedarg{y}*)[i] = (*\framedarg{beta}*) * (*\framedarg{y}*)[i];
						float temp[8];
						_mm256_storeu_ps(temp, vec_y);
						for (int k = 0; k < 8; ++k) {
							(*\framedarg{y}*)[i] += (*\framedarg{alpha}*) * temp[k];
						}
					}
				}
			\end{lstlisting}
		\end{subfigure}
		\hspace{1em}
		\begin{subfigure}[t]{0.45\textwidth}
			\centering
			\caption{FPGA Accelerator Pass}
			\label{lst:fpga}
			\begin{lstlisting}[language=C++]
				void fpga_gemv(float *(*\framedarg{A}*), float *(*\framedarg{x}*), float *(*\framedarg{y}*), float (*\framedarg{alpha}*), float (*\framedarg{beta}*), int (*\framedarg{lda}*), int (*\framedarg{N}*)) {
					// Transfer data to FPGA
					transfer_to_fpga((*\framedarg{A}*), (*\framedarg{x}*), (*\framedarg{y}*), (*\framedarg{alpha}*), (*\framedarg{beta}*), (*\framedarg{lda}*), (*\framedarg{N}*));
					
					// FPGA computation
					start_fpga_computation();
					
					// Retrieve results from FPGA
					retrieve_from_fpga((*\framedarg{y}*));
				}
			\end{lstlisting}
		\end{subfigure}
		\caption{Matrix-Vector Multiplication Implementations and FPGA Accelerator Pass}
		\label{fig:mvm_fpga_implementations}
	\end{figure}
	
\end{document}
